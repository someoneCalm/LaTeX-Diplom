\section*{ВВЕДЕНИЕ}
\addcontentsline{toc}{section}{ВВЕДЕНИЕ}

Современное управление проектами и командная работа претерпевают значительные изменения, обусловленные стремительным развитием цифровых технологий и необходимостью повышения эффективности рабочих процессов. Увеличение продуктивности команд и качества выполнения задач требует внедрения инновационных решений, способных минимизировать временные затраты, улучшить коммуникацию и оптимизировать распределение ресурсов. Одной из ключевых задач в этой области остается обеспечение прозрачности процессов, своевременное отслеживание прогресса и эффективное управление задачами в рамках проектов, особенно в условиях распределенных команд и многозадачности.

Традиционные методы управления проектами, зачастую основанные на ручном отслеживании, электронной почте или разрозненных инструментах, часто оказываются громоздкими, приводят к потере информации и затрудняют оперативное принятие решений. Современные подходы, такие как методология Kanban, в сочетании с веб-технологиями и интерактивными интерфейсами, предоставляют новые возможности для автоматизации и визуализации рабочих процессов, обеспечивая высокую наглядность и гибкость управления.

Создание удобных и интуитивно понятных веб-приложений для управления проектами на базе Kanban-досок открывает новые возможности для командной работы. Такие решения позволяют эффективно создавать, назначать и отслеживать задачи, визуализировать этапы их выполнения, управлять приоритетами и сроками, а также обеспечивать централизованное взаимодействие между участниками проекта. Интуитивно понятный интерфейс и автоматизация рутинных операций делают данный подход доступным и эффективным для широкого круга пользователей и типов проектов.

\textit{Цель данной работы} -- разработка веб-приложения для управления проектами по методологии Kanban, предоставляющего пользователям инструменты для эффективной организации, отслеживания и совместной работы над задачами. Для достижения этой цели необходимо решить следующие задачи:
\begin{enumerate}
	\item Определить требования к системе управления проектами и описать её функциональные возможности, включая управление пользователями, проектами, досками и задачами.
	\item Разработать архитектуру программного комплекса, включающую клиентскую и серверную части.
	\item Реализовать основной функционал системы, включая аутентификацию пользователей, создание и управление проектами, досками и задачами, а также механизм перетаскивания для изменения статуса и порядка задач.
	\item Разработать пользовательский интерфейс, обеспечивающий удобное и интуитивно понятное взаимодействие с системой.
	\item Провести комплексное тестирование системы для оценки её надежности, функциональности и удобства использования.
\end{enumerate}

\textit{Структура и объем работы}. Отчет состоит из введения, 4 разделов оcновной части, заключения, списка использованных источников, 2 приложений. Текст выпускной квалификационной работы равен \formbytotal{lastpage}{страниц}{е}{ам}{ам}.

\textit{Во введении} сформулирована цель работы, поставлены задачи разработки, описана структура работы, приведено краткое содержание каждого из
разделов.

\textit{В первом разделе} проводится анализ предметной области, изучение актуальных методологий управления проектами и перспективах их использования.

\textit{Во втором разделе} описываются требования к разрабатываемой системе управления проектами.

\textit{В третьем разделе} представлены проектные решения для программной системы, их описание и обоснование использования, а так же архитектура веб-приложения.

\textit{В четвертом} разделе приведены программные компоненты системы,
их описание и результаты тестирования.

\textit{В заключении} излагаются основные результаты работы, полученные в
ходе разработки.

\textit{В приложении А} представлен графический материал.

\textit{В приложении Б} представлены фрагменты исходного кода.