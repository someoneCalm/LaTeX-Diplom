\section*{ЗАКЛЮЧЕНИЕ}
\addcontentsline{toc}{section}{ЗАКЛЮЧЕНИЕ}

В ходе выполнения данной работы была разработана комплексная веб-система для управления проектами по методологии Kanban. Система реализована с использованием современных технологий веб-разработки, включая React для клиентской части, Node.js для серверной части и PostgreSQL в качестве СУБД.

Основные результаты работы:
\begin{enumerate}
\item Проведен анализ предметной области. Рассмотрены основные преимущества Kanban-методологии, проанализированы удобства ее применения.
\item Спроектирована и реализована архитектура системы. Разработана клиент-серверная архитектура, обеспечивающая четкое разделение логики представления и бизнес-логики.
\item Разработано веб-приложение с интуитивно понятным пользовательским интерфейсом. С использованием React созданы компоненты для отображения всех необходимых элементов для работы с профилем, проектами, досками и задачами, обеспечивая удобное взаимодействие пользователя с системой.
\item Проведено тестирование и отладка системы. Осуществлена проверка работоспособности основного функционала, включая API-эндпоинты, взаимодействие с базой данных, корректность отображения данных и пользовательского взаимодействия на клиентской стороне.
\end{enumerate}

Все основные требования, характерные для систем управления проектами Kanban-типа, были успешно реализованы. Разработанное приложение представляет собой масштабируемый и гибкий инструмент, предназначенный для эффективной организации командной работы, отслеживания прогресса выполнения задач и визуализации рабочих процессов. Система готова к дальнейшему развитию, включая возможное добавление новых функций и улучшений.

