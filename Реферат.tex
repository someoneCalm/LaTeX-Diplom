\abstract{РЕФЕРАТ}

Объем работы равен \formbytotal{lastpage}{страниц}{е}{ам}{ам}. Работа содержит \formbytotal{figurecnt}{иллюстраци}{ю}{и}{й}, \formbytotal{tablecnt}{таблиц}{у}{ы}{}, \arabic{bibcount} библиографических источников и \formbytotal{числоПлакатов}{лист}{}{а}{ов} графического материала. Количество приложений – 2. Графический материал представлен в приложении А. Фрагменты исходного кода представлены в приложении Б.

Перечень ключевых слов: программа, система, компьютер, веб-приложение, Agile, Kanban, доска, проект, задача, пользователь, пользовательский интерфейс, база данных, управление, компонент, модуль, сущность, React, Javascript, Node.js, Zustand, PostgreSQL.

Объектом разработки является программная система для управления проектами с использованием методологии Kanban.

Целью выпускной квалификационной работы является повышение эффективности управления проектами в IT-сфере и улучшения качества рабочего процесса путем разделения жизненного цикла проекта на подзадачи.

В процессе разработки веб-приложения были выделены основные сущности путем создания информационных блоков, использованы методы модулей, обеспечивающие работу с сущностями предметной области, а также корректную работу веб-приложения, разработан пользовательский интерфейс на основе React/CSS и фреймворка Node.js.

При разработке системы использовались библиотека React для клиентской части, фреймворк Node.js для серверной части, СУБД PostgreSQL для хранения данных о всех сущностях, а также язык Javascript и связанные с ним библиотеки.

Разработанная система была успешно протестирована.

\selectlanguage{english}
\abstract{ABSTRACT}

The scope of work is \formbytotal{lastpage}{page}{}{s}{s}. The work contains \formbytotal{figurecnt}{illustration}{}{s}{s}, \formbytotal{tablecnt}{table}{}{s}{s}, \arabic{bibcount} bibliographic sources, and \formbytotal{числоПлакатов}{sheet}{}{s}{s} of graphic material. The number of appendices is 2. Graphic material is presented in Appendix A. Source code fragments are presented in Appendix B.

List of keywords: program, system, computer, web application, Agile, Kanban, board, project, task, user, user interface, database, management, component, module, entity, React, Javascript, Node.js, Zustand, PostgreSQL.

The object of development is a software system for project management using the Kanban methodology.

The aim of the final qualifying work is to increase the efficiency of project management in the IT sphere and improve the quality of the work process by dividing the project life cycle into subtasks.

In the process of developing the web application, the main entities were identified by creating information blocks, module methods were used to ensure work with domain entities, as well as the correct operation of the web application, and a user interface was developed based on React/CSS and the Node.js framework.

During the system development, the React library was used for the client-side, the Node.js framework for the server-side, PostgreSQL DBMS for storing data about all entities, as well as the Javascript language and related libraries.

The developed system was successfully tested.
\selectlanguage{russian}