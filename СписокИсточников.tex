\addcontentsline{toc}{section}{СПИСОК ИСПОЛЬЗОВАННЫХ ИСТОЧНИКОВ}

\begin{thebibliography}{99}
	
	% --- Книги ---
	
	% I. Управление проектами и общие вопросы ИТ (10 книг)
	\bibitem{management1} Орлов, С. А. Основы управления ИТ-проектами / С. А. Орлов – Москва~: ДМК Пресс, 2023. - 320 с. – ISBN 978-5-9706-0881-4. – Текст~: непосредственный.
	
	\bibitem{management2} Семенов, М. И. Управление программными проектами: стандарты, инструменты, практика / М. И. Семенов, И. В. Волков – Санкт-Петербург~: БХВ-Петербург, 2024. - 450 с. – ISBN 978-5-9775-4321-7. – Текст~: непосредственный.
	
	\bibitem{management3} Ковалев, А. П. Риск-менеджмент в проектах разработки программного обеспечения / А. П. Ковалев – Москва~: Инфра-М, 2022. - 280 с. – ISBN 978-5-16-017854-9. – Текст~: непосредственный.
	
	\bibitem{management4} Липаев, В. В. Качество программного обеспечения: принципы и методы оценки / В. В. Липаев – Москва~: Янус-К, 2023. - 512 с. – ISBN 978-5-8037-0912-5. – Текст~: непосредственный.
	
	\bibitem{management5} Петров, К. Д. Эффективные коммуникации в ИТ-командах / К. Д. Петров – Москва~: Альпина Паблишер, 2024. - 198 с. – ISBN 978-5-9614-8234-1. – Текст~: непосредственный.
	
	\bibitem{management6} Вигерс, К. Разработка требований к программному обеспечению. 3-е издание / К. Вигерс, Дж. Битти – Москва~: БИНОМ. Лаборатория знаний, 2023. - 736 с. – ISBN 978-5-9909700-3-8. – Текст~: непосредственный.
	
	\bibitem{management7} Группа Акселос. ITIL® 4: Основы / Группа Акселос – Москва~: VAN HAREN PUBLISHING, 2024. - 320 с. – ISBN 978-9-40180-888-0. – Текст~: непосредственный.
	
	\bibitem{management8} Лебедев, А. Н. Лидерство в технологических командах: создание культуры инноваций и сотрудничества / А. Н. Лебедев – Санкт-Петербург~: Питер, 2023. - 288 с. – ISBN 978-5-4461-2345-6. – Текст~: непосредственный.
	
	\bibitem{management9} Ладушкин, В. С. Управление данными и Data Governance: стратегии и внедрение / В. С. Ладушкин – Москва~: ДМК Пресс, 2024. - 410 с. – ISBN 978-5-9706-1234-0. – Текст~: непосредственный.
	
	\bibitem{management10} Ершов, М. П. Основы облачных вычислений: технологии, модели и сервисы / М. П. Ершов – Москва~: Техносфера, 2023. - 350 с. – ISBN 978-5-94836-987-0. – Текст~: непосредственный.
	
	% II. Agile методологии (Общие) (5 книг)
	\bibitem{agile1} Кон, М. Agile: оценка и планирование проектов / М. Кон – Москва~: Вильямс, 2022. - 450 с. – ISBN 978-5-6045556-8-7. – Текст~: непосредственный.
	
	\bibitem{agile2} Вольф, К. С. Пользовательские истории: искусство гибкой разработки ПО / К. С. Вольф, Л. Гофман – Москва~: ДМК Пресс, 2023. - 304 с. – ISBN 978-5-9706-0901-9. – Текст~: непосредственный.
	
	\bibitem{agile3} Дерби, Э. Agile-ретроспектива: как превратить хорошую команду в великую / Э. Дерби, Д. Ларсен – Москва~: Эксмо, 2024. - 224 с. – ISBN 978-5-04-178834-3. – Текст~: непосредственный.
	
	\bibitem{agile4} Пуппол, М. Принципы Agile. Гибкое управление проектами для начинающих / М. Пуппол – Москва~: АСТ, 2023. - 128 с. – ISBN 978-5-17-154321-0. – Текст~: непосредственный.
	
	\bibitem{agile5} Сидоренко, В. П. Коучинг Agile-команд: практическое руководство / В. П. Сидоренко – Москва~: Доброе слово, 2024. - 280 с. – ISBN 978-5-9909876-5-4. – Текст~: непосредственный.
	
	% III. Kanban (5 книг)
	\bibitem{kanban1} Андерсон, Д. Канбан: альтернативный путь в Agile / Д. Андерсон – Москва~: Символ-Плюс, 2023. - 352 с. – ISBN 978-5-93286-234-5. – Текст~: непосредственный.
	
	\bibitem{kanban2} Бергманн, Б. Kanban изнутри: два года успешного применения / Б. Бергманн, П. Ахмар – Санкт-Петербург~: Питер, 2024. - 240 с. – ISBN 978-5-496-03456-7. – Текст~: непосредственный.
	
	\bibitem{kanban3} Леопольд, К. Практический Kanban: от идеи до поставки / К. Леопольд, З. Маурус – Москва~: ДМК Пресс, 2022. - 288 с. – ISBN 978-5-9706-0777-0. – Текст~: непосредственный.
	
	\bibitem{kanban4} Зайцев, М. В. Kanban для команд разработки ПО: внедрение и оптимизация / М. В. Зайцев – Москва~: Техносфера, 2023. - 192 с. – ISBN 978-5-94836-789-0. – Текст~: непосредственный.
	
	\bibitem{kanban5} Бенсон, Дж. Персональный Kanban: карта вашей работы / Дж. Бенсон, Т. де Мариа Бенсон – Москва~: Эксмо, 2024. - 208 с. – ISBN 978-5-04-165432-1. – Текст~: непосредственный.
	
	% IV. JavaScript и основы Frontend (5 книг)
	\bibitem{javascript1} Флэнаган, Д. JavaScript: подробное руководство. 7-е издание / Д. Флэнаган – Санкт-Петербург~: Символ-Плюс, 2022. - 720 с. – ISBN 978-5-93286-228-4. – Текст~: непосредственный.
	
	\bibitem{javascript2} Хавербеке, М. Выразительный JavaScript. 3-е издание / М. Хавербеке – Санкт-Петербург~: Питер, 2023. - 480 с. – ISBN 978-5-4461-1700-0. – Текст~: непосредственный.
	
	\bibitem{javascript3} Стефанов, С. JavaScript. Шаблоны / С. Стефанов – Москва~: Вильямс, 2024. - 272 с. – ISBN 978-5-8459-2299-9. – Текст~: непосредственный.
	
	\bibitem{javascript4} Сойер, К. Современный JavaScript для нетерпеливых / К. Сойер – Санкт-Петербург~: БХВ-Петербург, 2024. - 512 с. – ISBN 978-5-9775-1234-5. – Текст~: непосредственный.
	
	\bibitem{htmlcss1} Макфарланд, Д. Новая большая книга CSS / Д. Макфарланд – Москва~: Эксмо, 2023. - 640 с. – ISBN 978-5-04-112345-6. – Текст~: непосредственный.
	
	% V. React и Vite (5 книг)
	\bibitem{react1} Бэнкс, А. React и Redux: функциональная веб-разработка / А. Бэнкс, Е. Порселло – Санкт-Петербург~: Питер, 2023. - 336 с. – ISBN 978-5-4461-1987-5. – Текст~: непосредственный.
	
	\bibitem{react2} Фримен, А. React для профессионалов / А. Фримен – Москва~: ДМК Пресс, 2024. - 800 с. – ISBN 978-5-9706-0999-6. – Текст~: непосредственный.
	
	\bibitem{react3} Васильев, Д. А. React. Быстрый старт с хуками и TypeScript / Д. А. Васильев – Москва~: Солон-Пресс, 2023. - 352 с. – ISBN 978-5-91359-555-1. – Текст~: непосредственный.
	
	\bibitem{react4} Шварцмюллер, М. React – полное руководство (включая Hooks, React Router, Redux) / М. Шварцмюллер – Москва~: Эксмо, 2025. - 688 с. – ISBN 978-5-04-188765-4. – Текст~: непосредственный.
	
	\bibitem{vite1} Громов, И. П. Сборка современных веб-приложений с Vite: React, Vue, Svelte / И. П. Громов – Санкт-Петербург~: Наука и Техника, 2024. - 280 с. – ISBN 978-5-94387-888-9. – Текст~: непосредственный.
	
	% VI. Node.js и Backend (5 книг)
	\bibitem{nodejs1} Кантелон, М. Node.js. Разработка серверных приложений / М. Кантелон, Т. Хартер, Н. Раджмохан – Санкт-Петербург~: Питер, 2023. - 576 с. – ISBN 978-5-4461-1654-0. – Текст~: непосредственный.
	
	\bibitem{nodejs2} Чен, Ф. Разработка веб-приложений с помощью Node.js и Express / Ф. Чен – Москва~: Вильямс, 2024. - 432 с. – ISBN 978-5-907114-77-7. – Текст~: непосредственный.
	
	\bibitem{nodejs3} Тейшейра, П. Профессиональный Node.js: создание масштабируемых приложений / П. Тейшейра – Москва~: ДМК Пресс, 2022. - 624 с. – ISBN 978-5-9706-0811-1. – Текст~: непосредственный.
	
	\bibitem{nodejs4} Белов, А. В. Node.js: шаблоны проектирования и лучшие практики / А. В. Белов – Москва~: Бином. Лаборатория знаний, 2023. - 368 с. – ISBN 978-5-9963-6789-0. – Текст~: непосредственный.
	
	\bibitem{nodejs5} Голдберг, А. REST API на Node.js и MongoDB / А. Голдберг – Санкт-Петербург~: БХВ-Петербург, 2024. - 312 с. – ISBN 978-5-9775-3210-5. – Текст~: непосредственный.
	
	% VII. SQL и Базы Данных (5 книг)
	\bibitem{sql1} Карвин, Б. SQL. Антипаттерны: как избежать ошибок при проектировании баз данных / Б. Карвин – Санкт-Петербург~: Питер, 2023. - 320 с. – ISBN 978-5-4461-1543-3. – Текст~: непосредственный.
	
	\bibitem{sql2} Петров, А. И. SQL для анализа данных: практическое руководство / А. И. Петров – Москва~: ДМК Пресс, 2024. - 384 с. – ISBN 978-5-9706-1001-5. – Текст~: непосредственный.
	
	\bibitem{database1} Ульман, Дж. Д. Введение в системы баз данных / Дж. Д. Ульман, Дж. Уидом – Москва~: Вильямс, 2021. - 1072 с. – ISBN 978-5-8459-2153-3. – Текст~: непосредственный.
	
	\bibitem{postgres1} Дуглас, К. PostgreSQL для начинающих. От основ к профессиональной разработке / К. Дуглас – Москва~: Лори, 2024. - 480 с. – ISBN 978-5-85582-456-7. – Текст~: непосредственный.
	
	\bibitem{postgres2} Васкес, Л. PostgreSQL. Администрирование баз данных для профессионалов / Л. Васкес – Москва~: Символ-Плюс, 2022. - 704 с. – ISBN 978-5-93286-301-4. – Текст~: непосредственный.
	
	% VIII. Архитектура ПО и процессы разработки (2 книги)
	\bibitem{architecture1} Мартин, Р. Чистая архитектура: Искусство разработки программного обеспечения / Р. Мартин – Санкт-Петербург~: Питер, 2023. - 352 с. – ISBN 978-5-4461-0772-8. – Текст~: непосредственный.
	
	\bibitem{devops1} Ким, Дж. Проект «Феникс». Роман о том, как DevOps меняет бизнес к лучшему / Дж. Ким, К. Бер, Дж. Спэффорд – Москва~: Манн, Иванов и Фербер, 2022. - 480 с. – ISBN 978-5-00169-567-4. – Текст~: непосредственный.
	
	% --- Веб-сайты (9 источников) ---
	\bibitem{website_agilemanifesto1} Манифест гибкой разработки программного обеспечения : сайт / AgileManifesto.org. – [Б. м.] : AgileManifesto.org, 2001 – . – URL: \url{https://agilemanifesto.org/iso/ru/manifesto.html} (дата обращения: 17.04.2025). – Текст: электронный.
	
	\bibitem{website_kanbanuniversity1} Kanban University : сайт / Kanban University. – [Б. м.] : Kanban University, 2010 – . – URL: \url{https://kanban.university/} (дата обращения: 19.04.2025). – Текст: электронный.
	
	\bibitem{website_reactdocs1} React – официальная документация : сайт / React team. – [Б. м.] : Meta Platforms, Inc., 2013 – . – URL: \url{https://react.dev/} (дата обращения: 19.04.2025). – Текст: электронный.
	
	\bibitem{website_vitedocs1} Vite – официальная документация : сайт / Vite team. – [Б. м.] : Vite team, 2020 – . – URL: \url{https://vitejs.dev/guide/} (дата обращения: 19.04.2025). – Текст: электронный.
	
	\bibitem{website_nodejsdocs1} Node.js – официальная документация : сайт / OpenJS Foundation. – [Б. м.] : OpenJS Foundation, 2009 – . – URL: \url{https://nodejs.org/ru/docs/} (дата обращения: 19.04.2025). – Текст: электронный.
	
	\bibitem{website_postgresdocs1} PostgreSQL – официальная документация : сайт / The PostgreSQL Global Development Group. – [Б. м.] : The PostgreSQL Global Development Group, 1996 – . – URL: \url{https://www.postgresql.org/docs/} (дата обращения: 23.04.2025). – Текст: электронный.
	
	\bibitem{website_javascriptdocs1} JavaScript | MDN : сайт / Mozilla Developer Network. – [Б. м.] : Mozilla, 2005 – . – URL: \url{https://developer.mozilla.org/ru/docs/Web/JavaScript} (дата обращения: 18.04.2025). – Текст: электронный.
	
	\bibitem{website_habrarticle1} Канбан-метод в разработке: от теории к практике : сайт / Habr. – Москва : Хабр, 2023 – . – URL: \url{https://habr.com/ru/companies/otus/articles/780000/} (дата обращения: 17.04.2025). – Текст: электронный.
	
	\bibitem{website_smashingmagarticle1} Эффективное управление состоянием в React приложениях : сайт / Smashing Magazine. – Freiburg : Smashing Media AG, 2024 – . – URL: \url{https://www.smashingmagazine.com/tag/react/} (дата обращения: 19.04.2025). – Текст: электронный.
	
\end{thebibliography}
